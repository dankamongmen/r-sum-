\documentclass{article}
\usepackage{times}
\usepackage{fullpage}
\usepackage[small,compact]{titlesec}
\usepackage{textcomp}
\usepackage{url}
\usepackage{hyperref}
\nofiles

\newenvironment{tightitemize}
{\begin{itemize}
  \setlength{\itemsep}{1pt}
  \setlength{\parskip}{0pt}
  \setlength{\parsep}{0pt}}
{\end{itemize}}

%\setlength{\parindent}{0in}

\begin{document}
{\noindent}\small{\textbf{Nick Black
\href{mailto:nickblack@linux.com}{\nolinkurl{<nickblack@linux.com>}
\hspace*{\fill}\href{http://www.nick-black.com}{http://www.nick-black.com}}}}

{\noindent}\scriptsize{
855 Peachtree St NE, Atlanta, GA 30308
\hfill
US Citizen (b.\ 1980--10--05)
}

\nointerlineskip
\moveleft\hoffset\vbox{\hrule width\textwidth} 
\nointerlineskip

\pagestyle{empty}

\tiny
\begin{flushright}
{\noindent}
\textit{We work in the dark. We give what we have. We do what we can.}\\
\textit{Our doubt is our passion, and our passion is our task. The rest is the madness of art.}\\

\hfill--Henry James, \textit{The Middle Years}\end{flushright}
\footnotesize \subsection*{Objectives}
To continue researching, designing, implementing, deploying and maintaining
innovative solutions for keeping the bits flowing and cachelines humming.
To snatch order from a teeming chaos's complexity. To see the
fruits of my labors entrusted with the packetized dreams of millions.
To gaze back upon fifty-hour debugging sessions, seeing past the gnashed teeth, utter
fatigue, the too-long compiles, and \textit{know} the problem was solved. To---with
a little help from some good code---make the world a better place.\\

{\noindent}I am actively seeking consulting work in high-performance/scientific computing, systems programming, compiler design, or security.
\vspace{2mm}
\subsection*{Open Source \hfill\tiny\textit{See \href{http://www.nick-black.com/dankwiki/index.php/Hackery}{http://www.nick-black.com/dankwiki/index.php/Hackery} for more details.}}
I passionately believe in the many societal benefits of open source, and actively
participate in the Free Software community.
\begin{tightitemize}
\item Author and maintainer of numerous open source packages. Examples include:
\begin{tightitemize}
\item \textit{omphalos}, an application for securing and attacking local networks,
\item \textit{libtorque}, a multithreaded, architecture-adaptive event library, and
\item \textit{cubar} and \textit{libcudest}, reverse-engineering tools for NVIDIA's CUDA.
\end{tightitemize}
\item Contributed accepted patches to
the Linux kernel (Matrox driver, IPv4 stack, hugetlbfs), ZoL (ZFS on Linux),
Wireshark (IPv4 analysis),

{\indent} DynamoRIO, OProfile, Vim, Ncurses,
  strace, OpenSSL, APT-secure, Infinality, iproute2, bridge-utils, x86info, Snort, and other projects.
\item Maintained IBM's NGPT (Next Generation POSIX Threading) kernel patchset.
\end{tightitemize}

\vspace{2mm}
\subsection*{Employment and Education}
\begin{tightitemize}

\item \textbf{2012/01--Present: President/Principal Scientist, Sprezzatech (Atlanta, GA)\hfill \tiny{\textit{Founder}}}
\begin{tightitemize}
\item UNIX and HPC consulting (mainly custom software and hardware development).
\end{tightitemize}

\item \textbf{2011/01--2012/02: Senior Compiler Engineer, NVIDIA (Austin, TX)}
\begin{tightitemize}
\item Development on NVIDIA's OCG (Optimizing Code Generator) for Fermi and Kepler GPUs and Denver CPUs.
\item Led design and implementation work on a PTX/SASS/SL/Cg-unifying assembler.
\end{tightitemize}

\item \textbf{2010/05--2010/09: Chief Engineer, AccelerEyes LLC (Atlanta, GA)}
\begin{tightitemize}
\item Led design work for the libJacket\texttrademark\ CUDA primitives library. Assisted ongoing design of the Jacket\texttrademark\ MatLab\textsuperscript{\textregistered} plugin.
\item Implemented parallel GPU logical and numeric codes, including multidimensional convolutions, decompositions, sorts and searches.
\item Redesigned CPU-GPU memory transfer to use sharing, pinning and zero-copy, all atop a unified Bonwick-Alexandrescu allocator.
\end{tightitemize}

\item \textbf{2008/08--2010/05: Georgia Institute of Technology, Master of Science in Computer Science}
\begin{tightitemize}
\item Research project: ``Epicycles, Flywheels, and (Widening) Gyres: UNIX I/O
  Slouches Toward Multicore NUMA'' (2009). Explored the state of the art in
  large-scale, massively parallel servers, and how current APIs ought
  change. Results include the open \textit{libtorque} library and a 2010 paper,
  ``Portable Multithreaded Continuations for Scalable Event-Driven Programs''.
\item Teaching Assistant, ``High Performance Computer Architecture'' (CS6290---Superscalar, OOO, and manycore microarchitecture).\\
\end{tightitemize}

\item \textbf{2005/12--2009/11: Principal Engineer, McAfee (Alpharetta, GA)\hfill \tiny{\textit{Impact Engineer of the Year, 2007--2008}}}
\begin{tightitemize}
\item Sole backend developer for the Secure Web Proxy Service, a managed Web
  security/control system.
\item Worked with academia in anti-bot efforts, intimately studying and engaging
  the Storm botnet. Developed the \textit{argus panoptes} distributed anti-bot
  platform, and with it battled botnets across the IPv4 address space.
\item Lead developer of the IronNet\texttrademark\ appliance, making use of techniques
  including latent semantic analysis, clustering, and Markovian discrimination
  to prevent data leakage. This included the \textit{snare} ICAP server.
\item Assisted development of image spam detection in IronMail\texttrademark\ and
  reputation protection in PhishRegistry\texttrademark.
\item Developed \textit{garuda}, a CheckPoint\textsuperscript{\textregistered} OPSEC module interfacing with the
  TrustedSource\texttrademark\ reputation service.
\item Developed the \textit{HURLBAT} protocol testing tool, and SMTP, HTTP and ICAP modules.
\item Oversaw development of hardware, Knoppix\textsuperscript{\textregistered}-based bootable media, and custom tools for competitive field operations.
\item Rewrote the build system for IronMail\texttrademark\ and related products from scratch.
  Code review, automated building, virtualized testing, and unit testing were
  instituted along the way.
\item PCT/US2008/051869. Detecting Image Spam. 2008-07-31.\hfill\fbox{\textbf{US Patent}}\\
\end{tightitemize}

\pagebreak

\item \textbf{2000/08--2005/12: Senior Software Engineer, Reflex Security (Atlanta, GA)\hfill \tiny{\textit{Co-founder}}}
\begin{tightitemize}
\item Led research team, focused on intrusion prevention, parallelized and
  distributed intrusion detection, and multiple pattern matching. Implemented
  several techniques later published by academia. Led interviewing.
\item Sole developer of code for the Reflex Interceptor (now Reflex IPS\texttrademark\ and
  Reflex MG\texttrademark), a Layer-2 bridging NIPS running Linux. This included:
\begin{tightitemize}
    \item \textit{tako}, an IPS application statefully analyzing multiple GigE links inline
      and in real time, performing a forwarding verdict on each frame via use
      of \texttt{mmap(2)}ed packet sockets + custom \texttt{netlink(7linux)} messages, and
    \item \textit{geso}, an SMTP proxy making use of a Kaspersky\texttrademark\ Anti-Virus backend to
      filter mail inline and in real time, designed to be trivially extended
      to other store-and-forward protocols.
\end{tightitemize}
\item Assisted development of build systems, automated testing and benchmarking tools, and backend
      platform configuration management.
\item Assisted development of kernel patches to expand the netlink socket
      infrastructure and filter on \texttt{mmap(2)}-backed sockets.
\item PCT/US2004/023739. System and Method for Threat Detection and Response. 2005-02-03.\hfill \fbox{\textbf{US Patent}}\\
\end{tightitemize}

\item \textbf{1998/12--1999/12: Teaching Assistant, GT College of Computing (Atlanta, GA)}
\begin{tightitemize}
\item Recitations with \textasciitilde15 students, plus one-on-one meetings and grading of exams
and homeworks. Classes included:
\begin{tightitemize}
\item ``Programming Language Principles'' (CS 3411---Language design and comparative programming linguistics).
\item ``Models and Translation'' (CS 2330---Parsing, interpretation and compilation, virtual machines).
\item ``Control and Concurrency'' (CS 2430---Parallel computing, UNIX systems programming and ANSI/ISO C).
\item ``Instruction Set Architecture'' (CS 2760---Assembly language, stored programs, and architectural models).\\
\end{tightitemize}
\end{tightitemize}

\item \textbf{1998/09--2000/02, 2004/01--2005/05: Georgia Institute of Technology,
Bachelor of Science in Computer Science}
\begin{tightitemize}
\item Specializations in Theory, Systems, and Networking.
\item 2000 ACM Programming Team, GT Team A (``Gold'') (Captain).
\item 1999 ACM Programming Team, GT Team B (``White'') (Captain).
\item Research under Professor Wenke Lee (Topics: IDS, botnet models, anti-bot tech).
\end{tightitemize}
\end{tightitemize}

\vspace{2mm}
\subsection*{Awards and Distinctions}
\begin{tightitemize}
\item Credited corrections to Donald Knuth's \textit{The Art of Computer Programming}
   and George Varghese's \textit{Networking Algorithmics}.
\item Impact Engineer of the Year, 2007-2008 (McAfee / Secure Computing).
\item Faculty Honors, Georgia Tech 1999.
\item Dean's List, Georgia Tech 1998.
\item Third place, 1998 Questions Unlimited! National Academic Bowl Championships (New Orleans, LA).
\item Second place, 1997 Questions Unlimited! National Academic Bowl Championships (Chicago, IL).
\end{tightitemize}

\vspace{2mm}
\subsection*{Skills}
Algorithmic thinking. ANSI/ISO C. Development in the UNIX environment. Design
and analysis of algorithms. Network programming. Effective use and internals
of Linux, FreeBSD, their associated libc's and threading implementations,
GCC, GNU binutils, LLVM and the open source toolchain. Robust system design. 
Network security. Intrusion prevention. Establishment and detection of covert
channels. String algorithms. Applications of automata theory. Parallel
algorithms and design for multicore/manycore machines. Design
of programming languages. POSIX APIs along with Linux and FreeBSD
extensions thereof. Analysis of binaries. Most major network protocols and
their primary open source implementations. Effective use of cryptography. 
High-throughput, low-latency I/O models, scalable I/O and its implementation,
zero-copy networking, and network hardware design. Tools of network
enumeration and domination. x86 assembly language, including MMX and SSE. 
Computer architecture. Compiler design. Elegant system administration. 
Bayesian methods. Combinatorics, stochastics, topology, analysis, \textit{ad nauseam}. 

\vfill
\vbox{\hrule width\textwidth} 
\tiny
\begin{tightitemize}
\item\textbf{Operating Systems:} Linux (13 years), FreeBSD (6 years), Solaris (3 years), Windows NT (3 years)
\item\textbf{Languages (expert---minimum 10 active years):} C, Bourne shell, x86 and MIPS assembly, Prolog, Lisp, XSLT/XPath, GNU Make
\item\textbf{Languages (professional---minimum 5 active years):} C++, Python, Erlang, Haskell
\item\textbf{Languages (amateur---minimum 1 serious project):} SPARC, PTX, JVM and m68k assembly, Java, ML, Scheme, JavaScript (and XUL), Scala, Clojure
\item\textbf{Technologies:} GCC, LLVM, Pthreads, OpenMP, CUDA, OpenCL, Berkeley sockets, ACE, Flex/Bison, ANTLR, \LaTeX, Apache, Postfix, MediaWiki, DJBDNS, nginx, Bugzilla, VPNs, SSH, DocBook, GDB, valgrind, strace, IDAPro, Gecko/XULRunner, nmap, OProfile, perf, tcpdump, Pin, DynamoRIO, GnuTLS, OpenSSL, Avahi, iptables, Git, svn, Portsnap, ipfw, LARTC, socat, KVM/QEMU, Xen, VMWare ESXi, BLAS, ATLAS\ldots\\
If it can be done on a Linux box, I've probably done it.
\end{tightitemize}
\vbox{\hrule width\textwidth} 
\center{Last updated \today. The most recent version can be found on \href{http://nick-black.com/dankwiki/index.php/R%C3%A9sum%C3%A9}{my wiki}.
\LaTeX\ source is available at \href{https://github.com/dankamongmen/r-sum-}{GitHub}.\\
Copyright \textcopyright\ 2012 Nick Black. All rights reserved. Distributed under the \href{http://creativecommons.org/licenses/by-sa/3.0/}{Creative Commons Attribution-ShareAlike 3.0} license.}
\end{document}
