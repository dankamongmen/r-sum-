\documentclass{article}
\usepackage{times}
\usepackage{fullpage}
\usepackage[small,compact]{titlesec}
\usepackage{textcomp}
\usepackage{url}
\usepackage{hyperref}
\nofiles

\newenvironment{tightitemize}
{\begin{itemize}
  \setlength{\itemsep}{1pt}
  \setlength{\parskip}{0pt}
  \setlength{\parsep}{0pt}}
{\end{itemize}}

%\setlength{\parindent}{0in}

\begin{document}
{\noindent}\small \textbf{Nick Black
\href{mailto:nickblack@linux.com}{\nolinkurl{<nickblack@linux.com>}}
\hfill \href{http://dank.qemfd.net}{http://dank.qemfd.net}}

\nointerlineskip
\moveleft\hoffset\vbox{\hrule width\textwidth} 
\nointerlineskip

\tiny
\begin{flushright}
{\noindent}
\textit{We work in the dark. We give what we have. We do what we can.}\\
\textit{Our doubt is our passion, and our passion is our task. The rest is the madness of art.}\\

\hfill--Henry James, \textit{The Middle Years}
\end{flushright}
\footnotesize
\subsection*{Objectives}
To continue researching, designing, implementing, deploying and maintaining
innovative solutions for keeping the bits flowing and cachelines humming.
To snatch order from a teeming chaos in the least code possible. To see the
fruits of my labors entrusted with the packetized dreams of millions. To feel I
owe the world each day's improvements to my source trees. To gaze back upon
sixty-hour debugging sessions, seeing past the gnashed teeth, the utter
fatigue, the too-long compiles, and \textit{know} that the problem was solved.

\vspace{2mm}

\subsection*{Open Source \hfill\tiny\textit{(See \href{http://dank.qemfd.net/dankwiki/index.php/Hackery}{http://dank.qemfd.net/dankwiki/index.php/Hackery} for more details)}}
\begin{tightitemize}
\item Authored numerous open source packages, including:
\begin{tightitemize}
\item \textit{libtorque}, a multithreaded, architecture-adaptive event library
\item \textit{cubar} and \textit{libcudest}, NVIDIA CUDA reverse-engineering tools
\item \textit{ptracer}, an instruction trace generator
\end{tightitemize}
\item Contributed accepted patches to Wireshark's core IPv4 procesing, several Linux components
  (Matrox driver, IPv4 stack, and hugetlb filesystem), DynamoRIO,
  strace, OpenSSL, APT-secure, iproute2, bridge-utils, and Snort.
  Maintained IBM's NGPT (Next Generation POSIX Threading) kernel patchset.
\end{tightitemize}

\subsection*{Employment and Education}
\begin{tightitemize}
\item \textbf{2010/08--Present: Georgia Institute of Technology,
Doctor of Philosophy in Computer Science
\hfill \tiny\textit{(on indefinite hiatus)}}
\footnotesize
\begin{tightitemize}
\item Investigated architecture, operating system, language and compiler support for high-performance network servers.
\item Advised by Professor Tom Conte as part of the \href{http://comparch.gatech.edu/}{\texttt{comparch}} and \href{http://tinker.cc.gatech.edu/}{TINKER} research groups.
\end{tightitemize}

\item \textbf{2010/05--2010/09: Chief Engineer, AccelerEyes LLC (Atlanta, GA)}
\begin{tightitemize}
\item Led design work for the ``Jacket'' MatLab\textregistered\ plugin, and AccelerEyes's core library of CUDA functionality.
\item Implemented and analyzed GPU logical and numeric codes, including multidimensional convolutions, decompositions, sorting and searching.
\item Redesigned CPU-GPU memory transfer to use sharing, pinning and zero-copy.
\item Reimplemented build system, autotesting, and office network from scratch.
\end{tightitemize}

\item \textbf{2008/08--2010/05: Georgia Institute of Technology, Master of Science in Computer Science}
\begin{tightitemize}
\item Research project: ``Epicycles, Flywheels, and (Widening) Gyres: UNIX I/O
  Slouches Toward Multicore NUMA'' (2009). Explored the state of the art in
  large-scale, massively parallel servers, and how current APIs ought
  change. Results include the open \textit{libtorque} library and a 2010 paper,
  ``Portable Multithreaded Continuations for Scalable Event-Driven Programs''.
\item Sixteen classes, with twelve from among Systems and Information Security.
\item Teaching Assistant (CS6290, ``High Performance Computer Architecture'').
\end{tightitemize}

\item \textbf{2005/12--2009/11: Principal Engineer, McAfee (Alpharetta, GA)}
\begin{tightitemize}
\item Impact Engineer of the Year, 2007--2008.
\item Sole backend developer for the Secure Web Proxy Service, a managed Web
  security/control system.
\item Worked with academia in anti-bot efforts, intimately studying and engaging
  the Storm botnet. Developed the argus panoptes distributed anti-bot
  platform, and with it battled botnets across the IPv4 address space.
\item Lead developer of the IronNet\texttrademark\ appliance, making use of techniques
  including latent semantic analysis, clustering, and Markovian discrimination
  to effect what IDC calls ``compliance''. This included the snare ICAP server.
\item Developed algorithms to detect image spam used by SecureMail\texttrademark, and
  reputation protection systems used by the PhishRegistry\texttrademark\ service.
\item Developed garuda, a CheckPoint\textsuperscript{\textregistered} OPSEC module interfacing with the
  TrustedSource\texttrademark\ reputation service.
\item Developed the hurlbat protocol testing tool, and SMTP, HTTP and ICAP modules.
\item Planned and led execution of IPv6 integration into Secure Computing products.
\item Developed a Knoppix\textsuperscript{\textregistered}-based CD for competitive operations in the field.
\item Collaboratively designed the architecture and protocols behind a new
  implementation of the TrustedSource\texttrademark\ reputation service.
\item Rewrote the build system for IronMail\texttrademark\ and related products from scratch.
  Code review, automated building, virtualized testing, and unit testing were
  instituted along the way. Converted crufty CVS repositories to shiny
  Subversion repositories and investigated distributed source control.
\item Played a leading role in interviews of new research candidates.
\item PCT/US2008/051869. Detecting Image Spam. 2008-07-31.
\end{tightitemize}

\pagebreak

\item \textbf{2000/08--2005/12: Senior Software Engineer, Reflex Security (Atlanta, GA)}
\begin{tightitemize}
\item Led research team, focused on intrusion prevention, parallelized and
  distributed intrusion detection, and multiple pattern matching. Implemented
  several techniques later published by academia. Led interviewing.
\item Sole developer of code for the Reflex Interceptor (now Reflex IPS\texttrademark\ and
  Reflex MG\texttrademark), a Layer-2 bridging NIPS running Linux. This included:
\begin{tightitemize}
    \item \textit{tako}, an IPS application statefully analyzing multiple GigE links inline
      and in real time, performing a forwarding verdict on each frame via use
      of \texttt{mmap(2)}ed packet sockets + custom \texttt{netlink(7linux)} messages,
    \item \textit{geso}, an SMTP proxy making use of a Kaspersky Anti-Virus backend to
      filter mail inline and in real time, designed to be trivially extended
      to other store-and-forward protocols,
    \item build systems, automated testing and benchmarking tools, and backend
      platform configuration management for these applications, and
    \item kernel patches to change driver mechanics, expand the netlink socket
      infrastructure, and provide an interface into \texttt{mmap(2)}-backed sockets.
\end{tightitemize}
\item PCT/US2004/023739. Sys+Method for Threat Detection and Response. 2005-02-03.
\end{tightitemize}

\item \textbf{1998/12--1999/12: Teaching Assistant, GT College of Computing (Atlanta, GA)}
Recitations with ~15 students, plus one-on-one meetings and grading of exams
and homeworks. Classes included:
\begin{tightitemize}
\item ``Programming Language Principles'' (CS 3411---Language design and comparative programming linguistics).
\item ``Models and Translation'' (CS 2330---Parsing, interpretation and compilation, virtual machines).
\item ``Control and Concurrency'' (CS 2430---Parallel computing, UNIX systems programming and ANSI/ISO C).
\item ``Instruction Set Architecture'' (CS 2760---Assembly language, stored programs, and architectural models).
\end{tightitemize}

\item \textbf{1998/09--2000/02, 2004/01--2005/05: Georgia Institute of Technology,
Bachelor of Science in Computer Science}\\
Specializations in Theory, Systems, and Networking
\begin{tightitemize}
\item ACM Programming Team, 2000, GT Team A (Captain), 1999 GT Team B.
\item Research under Wenke Lee (Topics: IDS, botnet models, anti-bot tech).
\end{tightitemize}

\end{tightitemize}

\subsection*{Awards and Distinctions}
\begin{tightitemize}
\item Credited corrections to Donald Knuth's \textit{The Art of Computer Programming}
   and George Varghese's \textit{Networking Algorithmics}.
\item Impact Engineer of the Year, 2007-2008 (McAfee / Secure Computing).
\item Faculty Honors, Georgia Tech 1999.
\item Dean's List, Georgia Tech 1998.
\item Third place, 1998 Questions Unlimited! National Tournament (Walton HS).
\item Second place, 1997 Questions Unlimited! National Tournament (Walton HS).
\end{tightitemize}

\subsection*{Skills}
Algorithmic thinking. ANSI/ISO C. Development in the UNIX environment. Design
and analysis of algorithms. Network programming. Effective use and internals
of Linux, FreeBSD, their associated libc's and threading implementations,
GCC, GNU binutils, LLVM and the open source toolchain. Robust system design. 
Network security. Intrusion prevention. Establishment and detection of covert
channels. String algorithms. Applications of automata theory. Parallel
algorithms and design for multicore/manycore machines. Principles and design
of programming languages. POSIX APIs, along with FreeBSD, Solaris and Linux's
extensions thereof. Analysis of binaries. Most major network protocols and
their primary open source implementations. Effective use of cryptography. 
High-throughput, low-latency I/O models, scalable I/O and its implementation,
zero-copy networking, and network hardware design. Tools of network
enumeration and domination. x86 assembly language, including MMX and SSE. 
Computer architecture. Compiler design. Elegant system administration. 
Bayesian methods. Combinatorics, stochastics, topology, analysis, \textit{ad nauseam}. 

\vfill
\begin{tightitemize}
\item \textbf{Operating Systems:} Linux (12 years), FreeBSD (6 years), Solaris (3 years)
\item \textbf{Languages (expert, minimum 10 years):} C, Prolog, Lisp, Bourne shell, x86 and MIPS assembly, XSLT/XPath
\item \textbf{Languages (professional, minimum 5 years):} C++, Python, Erlang, Haskell
\item \textbf{Languages (amateur, minimum 1 serious project):} SPARC and M68k assembly, Java, ML, Scheme, JavaScript (and XUL), Scala
\item \textbf{Technologies:} GCC, LLVM, Pthreads, OpenMP, CUDA, OpenCL, Berkeley sockets, ACE, Boost, Flex/Bison, ANTLR, \LaTeX, Apache, Postfix, MediaWiki, Bugzilla, VPNs, SSH, ERESI, GDB, valgrind, Gecko/FireFox/XULRunner, nmap, tcpdump, Pin, DynamoRIO, vim
\end{tightitemize}
\end{document}
